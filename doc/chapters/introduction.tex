\chapter{Introduction}
\label{chapter:introduction}

Nowadays, computing science intertwines with various fields of studies posing new challenges for the people of the industry. Every aspect of science is dominated by the technology, just like the everyday lives across the world.

A good example of such interdisciplinary connection is a situation when the companies' managers are required to answer a difficult optimisation question. These businesses mostly are made of different branches, which require transferring of goods between them. Each element of the product could be manufactured in a various departments and is needed in the last part of the assembly in the main office. It could be easier when they produce the whole product in one location, yet seldom it is better to divide the responsibility for specialists in each sector. Although, this process generates som extra costs for the company, it would be crucial to minimise the expenses during assignment of the branches to the locations. Trying to resolve this by hand could be a long process due to the complexity of the problem. However, with usage of computers, it could be answered in a shorter time. The result may not be the optimal one in each case, but it should meet most given assumptions, which is enough to put it into real life usage.

In computer science, this dilemma can be stated as the Quadratic Assignment Problem (QAP).  The QAP is a combinatorial optimisation problem, which was defined by Koopmans and Beckmann in 1957 \cite{koopmans-beckmann1957}, and is a generalisation of an assignment problem. It is NP-hard thus even for small instances finding results is done by approximate methods.

The QAP can be formulated as follows: Given $ n $ different facilities ($F$) and $ n $ different locations ($L$), a weight function $ w: F \times F \mapsto R $ between facilities and a distance function $ d: L \times L \mapsto R $ between locations, find the assignment minimising this cost function:

\begin{equation}
min \sum_{a, b \in F} w(a, b) * d( f(a), f(b))
\end{equation}

Over the years, a number of methods for solving this problem were established. Nevertheless, there is still a place for improvement and experts are searching for new ways of resolving that. Inspired by their works this thesis tries to find an innovative method using living organisms.

The \textit{Physarum polycephalum}, also known as a true slime mould, is a plasmodial organism of yellow colour. Its single cell body is considered to be the biggest in the world \cite{stephenson1994myxomycetes}. Taking into account the current state of classification, it belongs to the Kingdom Protista, however, this used to frequently change due to the fact that it does not exactly match any of the recognised kingdoms. The \textit{Physarum polycephalum} move very slowly, in a branching pattern as it forages for new food sources. It ingests bacteria, fungal spores and during the experiments --- oatmeals.

\section{Motivation}
\label{section:introduction_motivation}


\section{Goal}
\label{section:introduction_goal}

This thesis presents the road to solving Quadratic Assignment Problem (QAP) based on behaviour of Physarum machines.
Deep analysis of each part of the main dilemma is essential to proceed with future works.

The first task is to carry out the detailed investigation of the behaviour and capabilities of the slime mould. Without the understanding of organisms, it is not possible to replicate its operations. For this purpose, the living plasmodia will be observed, giving us details of the schemes of the movement when looking for food. This will be studied in order to extract similar patterns and facilitate the creation of
calculation method based on them, which could be transported into the computer environment. We will implement such plasmodial behaviour for solving QAP.

Furthermore, not only the direct observation of \textit{Physarum polycephalum} behavior is needed here, but also a careful examination of previous studies. Such research will also give as a thorough look into behaviour and abilities of the slime mould, especially taking into account the possibility of using our model algorithm as an inspiration for solving computational problems.

Next, the analysis of the research related to the QAP in general will be conducted, leading to better understanding of the problem and showing the current practices for solving it. Recognising the dilemma will make it easier to fit the algorithm based on slime moulds to the QAP.

The key element of this thesis is to apply methods used by \textit{Physarum polycephalum} for solving QAP. This step will consist of adaptation of the biological mechanisms, implementation of simulation and analysis of the results. In this way, we connect the previously acquired theoretical knowledge and a practical task. And last, but not least, our aim will be to create the innovative method for solving QAP.


\section{Chapters}
\label{section:introduction_chapters}

The thesis is divided into five chapters and includes one appendix.

\begin{itemize}  
\item Chapter two describes the physarum organisms characteristics such as a position in the hierarchy, basic information about the species, basics of operations, emerging behavior and previous research.
\item Chapter three outlines the quadratic assignment problem (QAP). It consists of a different interpretation, practical usages, current exact solution and current heuristic. 
\item Chapter four presents the algorithm, which will be proposed as the result of this thesis. It will be a pseudophysarum machine providing working metaheuristics based on observed behavior.
\item Chapter five summarizes the research and is focusing on future work ideas.
\item Appendix A includes description of hardware-software platform, which is used for examination of physarum.
\end{itemize}

\section*{Work Distribution}
\label{section:introduction_distribution}

The work was distributed between two coworkers in a manner described below:
\begin{itemize}
  \item Amadeusz Juskowiak conducted a research about Physarum Polycephalum and he observed the behavior of alive organisms. He also designed and implemented the algorithm approximating QAP.
  \item Wioletta Różańska conducted a research about quadratic assignment problem. She also implemented the algorithm approximating QAP and carried out comparative tests with existing solutions.
\end{itemize}

Next to these tasks both authors wrote the text of this thesis.

