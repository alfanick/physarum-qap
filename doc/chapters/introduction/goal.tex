\section{Goal}
\label{section:introduction_goal}

This thesis presents the road to solving Quadratic Assignment Problem (QAP) based on behaviour of physarum machines. In order to reach meaningful conclusions, it is needed to analyse deeply each part of the main dilemma.

The first task is carrying out the detailed investigation of the behaviour and capabilities of the slime mould. Without the understanding of organisms, it is not possible to replicate its operations. For this purpose, the living plasmodia will be observed, giving us details of the schemes how it moves when looking for food. This will be studied in order to extract similar patterns and facilitate the creation of their calculation method, which could be transported into the computer environment. We will implement such plasmodial behaviour for solving QAP.

Furthermore, not only the direct observation of their behavior is needed here, but also a careful examination of previous studies. Such research will give as a thorough look into behaviour and abilities of the slime mould.

Next, the analysis of the research related to the QAP will be conducted, leading to better understanding of the problem and showing the current practices for solving it. Recognising the dilemma will make it easier to fit algorithm based on slime moulds to the QAP.

The key element of this thesis is applying physarum methods for solving QAP. This step will consist of adapting the mechanisms, implementation of simulation and reading its results. It will summarise the previously acquired theoretical knowledge in a practical task.

And last, but not least, our aim will be to create the innovative method for solving QAP.
