\section{Implementation}
\label{section:project_implementation}

The algorithm is implemented using C++ programming language, which creates minimal overhead and exposes power of underlying hardware. The language has been chosen for its performance, built-in rich standard library and because of the authors experience with this language. The program has a command-line interface which can be easily integrated in the bigger workflow.

The code source closely resembles algorithm in its essential explore, crawl and merge phases. The codebase has been divided into functional units such as \texttt{Experiment} describing \texttt{Environment} where \texttt{Plasmodium} can feed on \texttt{Solution} to instance of the \texttt{Problem}. The code is comprehensively commented --- one can read the source files for further details of the implementation.

\subsection{Building instructions}

In order to build the program a POSIX-compatible environment is required (i.e. Mac OS X, Linux), with minor changes it should be compatible with Windows platforms. Tools such as C++11 compiler (authors used \texttt{clang 703.0.29} and \texttt{g++ 4.9.4}) and $cmake$ build tool are required. The executable can be created using \texttt{build.sh} command, which will create two binary files --- \texttt{bin/physarum} and \texttt{bin/physarum-debug}, where the second one genereates detailed machine readable logs of plasmodial behaviour.

\subsection{Data format}

The program uses input format of QAPLIB \cite{burkard1997qaplib} --- first line contains size of the problem $n$, followed by $n{\times}n$ numbers representing distance matrix, followed by another $n{\times}n$ integers representing the weight matrix. An output is given to the standard output in format of an integer representing the problem size $n$, followed by cost of the solution $f$, followed by $n$ numbers representing the proposed assignment.

\subsection{Configuration}

The program accepts options in format of flags. The configuration options closely resemble parameters described previously in section \ref{subsection:am_parameters}. Complete list of flags with default values is provided in table~\ref{table:pi_options}. Some short description of the options can be always shown when executed with \texttt{-h} or \texttt{--help} flag.

\begin{table}[h]
  \centering
  \caption{Available configuration options for \texttt{physarum}}
  \label{table:pi_options}
  \begin{tabularx}{\textwidth}{r|r|c|X}
    Option name & Short name & Default value & Description \\ \hline \hline
    \texttt{--data} & \texttt{-d} & \textbf{none} & Required path to problem definition file \\ \hline
    \texttt{--samples} & \texttt{-k} & $10$ & Number of sampled solutions \\ \hline
    \texttt{--population} & \texttt{-l} & $1$ & Number of plasmodia in a colony \\ \hline
    \texttt{--e\_initial} & \texttt{-i} & $0.0$ & Initial energy for each plasmodium ($E_{initial}$) \\ \hline
    \texttt{--e\_explore} & \texttt{-e} & $0.001$ & Exploration energy ($E_{explore}$) \\ \hline
    \texttt{--e\_crawl} & \texttt{-c} & $0.001$ & Crawling energy ($E_{crawl}$) \\ \hline
    \texttt{--scale} & \texttt{-a} & $0.1$ & Scaling factor for food transformation ($a$) \\ \hline
    \texttt{--base} & \texttt{-q} & $10.0$ & Exponential base for food transformation ($q$) \\ \hline
    \texttt{--no-merge} & \texttt{-m} & $false$ & Disables merge phase \\ \hline
    \texttt{--time} & \texttt{-t} & $30$ & Time (in seconds) for limiting simulation \\ \hline
    \texttt{--seed} & \texttt{-s} & $-1$ & Number used for seeding RNG ($-1$ for time based value), can be used for repeating experiments \\ \hline \hline
  \end{tabularx}
\end{table}

\subsection{External libraries}

The implementation uses \texttt{Flags.hh} library by Song Gao for parsing command line options and \texttt{getRSS.c} by David Robert Nadeau for analysing memory usage when collecting logs. Both libraries are not essential parts of the implementation, their source codes are includes in \texttt{src/external/} with adequate licenses and copyright informations. 

