\chapter{Conclusion}
\label{chapter:conclusion}

The goal of this thesis was to research a living organism \textit{Physarum Polycephalum}, gain an insight into Quadratic Assignment Problem field and propose a novel algorithm for approximating such problem. Every aspect of the goal has been an educative experience.

The first challange was doing a thorough research on the slime mould. He have read multiple works, which has given as a knowledge about behaviour of the organism and current state of its emerging computional capabilities. Some of these facts were later confirmed by observation of a real living plasmodium. Simultaneously we explored the area of Quadratic Assignment Problem --- its usecases, various definitions and possible algorithms. With such complex optimisation problem, often approximate methods are prefered over the exact ones. These methods give good enough results within reasonable short time.

Initially an idea for a physarum machine was being considered. It would be a compound system, made of complex mathematical transformations and methods of observation a living \textit{Physarum Polycephalum}, but as a result such machine would work only on small impractical instances of the problem. Eventually a novel algorithm based on observed behaviour of the slime mould has been designed. 

The Physarum-based Metaheuristic algorithm is made of four essential parts: solution to energy transformation, initial sampling, exploration and crawling phase. Design of these phases has been inspired by structures and movements made by the crawling plasmodium. The algorithm explores a space search of the problem by randomised fashion, while doing a solution optimisation and avoiding being stuck in a local minima. The algorithm uses concept of energy, which is heavily parametrised and can be tuned resulting in very different behaviour of a virtual plasmodia.

In order to perform tests, a model implementation of the algorithm has been made. As the algorithm depends on multiple configuration parameters, a crucial part of the thesis was rigorous testing. We tested behaviour of the algorithm on a variety of testcases from recognizable QAPLIB. During the testing phase an improvement to the algorithm has been made, which did not change a behaviour of the simulated slime mould, but changed a method of acquiring the result. This improvement resulted in giving a satisfactory assignments.

The implementation has been tested on multiple test cases, often yielding an optimum assignment. Moreover some of the results have been characterised by an optimal cost, but the assignments were different than ones proposed in the literature. In comparison to already existing algorithms, the Physarum-based Metaheuristic produces a competitive solutions. On compared testcases, our algorithm works better than popular \textit{GRASP}, simulated annealing or ant colony methods, with only much more complex \textit{ANGEL} preceding it.

In the end we are pleased with the outcome of this thesis --- we have gained knowledge on world of unconventional computing, challenged a complex optimisation problem, designed an unique algorithm on par with leading solutions. We encourage everyone to have a glimpse into world of nature and take inspiration from its organic behaviour to solve everyday problems.


\section*{Future work}

While working on this thesis, we have come with ideas for further research and possible extension of our work. Definitely, the topic of unconventional computing using \textit{Physarum Polycephalum} still fascinates us, one should explore their emerging capabilities to further level, even by creation a real world, physical physarum machine. On the other hand, the Physarum-based Metaheuristic algorithm could have been reimplemented on GPU devices, exploiting their power of massive parallel computations --- the algorithm is population based, which can be easily distributed across such device. However, each virtual plasmodium share the same environment with its food sources which could cause synchronisation problems. Moreover a hyperheuristic method can be designed in order to simplify tuning the parameters of the algorithm. This would be a real help when dealing with large QAP instances.

More research can be done on using Physarum-based Metaheuristic with other optimisation problems, some attempts to solve TSP are presented in Appendix \ref{chapter:tsp}, however results are preliminary and more work should be done.
