\section{Goal}
\label{section:introduction_goal}

This thesis presents the road to solving quadratic assignment problem (QAP) using physarum machines. In order to reach meaningful conclusions, it is needed to analyse deeply each part of the main dilemma.

The first task is to carry out the detailed investigation of the behavior and capabilities of physarum. Without the understanding of organisms, it is not possible to replicate its operations. For this purpose, the living physarums will be observed and described, which will mainly consist of the schemes of ways of moving to find food. This will be studied in order to extract similar patterns and facilitate the creation of their calculation method, which could be transported into the computer environment. Additionally, it will determine whether they fit into QAP.

Furthermore, not only the direct observation of their behavior is needed here, but also a careful examination of previous studies. It will show already discovered characteristics, which could have been unnoticed on our own research.

Next, the analysis of the research related to the QAP will be required leading to better understanding of the problem and showing the current practices for resolving it. Recognising the dilemma will make it easier to fit algorithm based on slime moulds to the QAP.

The key element of this thesis is applying physarum methods for solving QAP. This step will consist of adapting the mechanisms, implementation of simulation and reading its results. It will summarise the previously acquired theoretical knowledge in a practical task.

And last, but not least, our aim will be to create the innovative method for solving QAP.
