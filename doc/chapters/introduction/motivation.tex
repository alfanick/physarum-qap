\section{Motivation}
\label{section:introduction_motivation}

The motivation for this thesis was indirect interest in topics related to computer science, but also of the world around us. The behavior of physarum, which is often compared to a simple machine, creates many opportunities to unveil a biological side of computer science making the topic fascinating.

Nowadays, scientists put great emphasis on discovering and analysing the nature. It is done to improve the world surrounding us. Generally, two ways of development of this field of study could be distinguished. The first one focuses on improving the biological flaws of humans and animals. A good case is studies related to the creation of natural prosthetics that make life easier for the people without limbs. The second one is a transmission of the known naturally patterns to the computing environment. Observing the nature leads to logical algorithms, which usage can solve issues seemingly unrelated to originally presented problems and often gives much better solution than working it out greedy. For example, thanks to such research, the ant algorithm was implemented, which shows the behavior of an ant colony searching for the best path between their home and food. These unconventional methods of inventing algorithms allow excellent results for hitherto very complicated mathematical problems.

The physarums have great potential in both, the first and the second case. Until now, several studies linked to these organisms were conducted, though, it still remains a mystery to many experts. One example of an experiment carried on slime moulds was solving the maze. The organism found the shortest path between two oatmeals in the environment with walls, thus finding the solution for the maze. More detailed description and more cases are presented in the later chapter of this thesis. Nonetheless, these interesting achievements were the reason behind the choice of the subject.

Also, the thesis focuses on the quadratic assignment problem, which is a challenging topic. It reflects the real difficulties faced by the managers of logistics companies. They need optimal results, however, the complexity of the dilemma makes it almost impossible to resolve in a reasonable time. This demand urges scientists to explore this issue further and try to look for a reliable way of solving the problem. 

The QAP could be a great challenge for inventing a new unconventional algorithm based on the behavior of physarum. 