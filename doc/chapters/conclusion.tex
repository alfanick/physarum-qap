\chapter{Conclusion}
\label{chapter:conclusion}

The goal of this thesis was to research a living organism \textit{Physarum polycephalum}, gain an insight into Quadratic Assignment Problem field and propose a novel algorithm for approximating such problem. Every aspect of the goal has been an educative experience.

The first challange was a thorough research on the slime mould. We have read multiple works, which gave us a knowledge about behaviour of the \textit{Physarum polycephalum} and the current state of its emerging computional capabilities. Some of these facts were later confirmed by observation of a real living plasmodium done during preparation of this thesis. Simultaneously we explored the area of Quadratic Assignment Problem --- its usecases, various definitions and possible algorithms. With such a complex optimisation problem, often approximate methods are prefered over the exact ones. These methods give results good enough within reasonable short time.

Initially an idea for a physarum machine was considered. It would be a compound system, made of complex mathematical transformations and methods of observation a living \textit{Physarum polycephalum}. However, as a result such machine would work only on small impractical instances of the problem. Eventually a novel algorithm based on the observed behaviour of the slime mould has been designed. The Physarum-based Metaheuristic algorithm is made of four essential parts: an initial sampling, solution to energy transformation, exploration and crawling phase. Design of these phases has been inspired by structures and movements made by the foraging plasmodium. The algorithm explores a space search of the problem in a randomised fashion, while doing an optimisation of a solution and avoiding being stuck in a local minima. The algorithm uses concept of energy, which is heavily parametrised and can be tuned resulting in very different behaviour of a virtual plasmodia.

In order to perform tests, a model implementation of the algorithm has been made. As the algorithm depends on multiple configuration parameters, a crucial part of the thesis was a rigorous testing. We tested the behaviour of the algorithm on a variety of testcases from recognizable QAPLIB. During the testing phase an improvement to the algorithm has been made, which did not change a behaviour of the simulated slime mould, but changed a method of acquiring the result. This improvement resulted in giving a satisfactory assignments.

The implementation has been tested on multiple test cases, often yielding an optimum assignment. Interestingly, even if some of the results have been characterised by an optimal cost, the assignments were different than ones proposed in the literature. In comparison to already existing algorithms, the Physarum-based Metaheuristic produces competitive solutions. On compared testcases, our algorithm works better than popular \textit{GRASP}, simulated annealing or ant colony methods, with only much more complex \textit{ANGEL} method preceding it.

To conclude, our work explores field of unconventional computing, challenges a complex optimisation problem and proposes a design of unique metaheuristic algorithm on a par with leading solutions. We took an inspiration from the world of nature and created this novel method of computation. Even though the design of the algorithm looks simple, its emerging properties make it useful for solving the practical cases. 


\section*{Future work}

During the work on this thesis, we have come with ideas for a further research and possible extensions of our work. The topic of unconventional computing using \textit{Physarum polycephalum} is a fascinating area of research that urges for an exploration of its capabilities to further level, even by creation of a real world, physical physarum machine. What is more, the Physarum-based Metaheuristic algorithm could be reimplemented on GPU devices, exploiting their power of massive parallel computations --- the algorithm is population based, which can be easily distributed across such devices. However, each virtual plasmodium share the same environment with its food sources which could cause synchronisation problems. Moreover a hyperheuristic method can be designed in order to simplify tuning the parameters of the algorithm. This would be a real help when dealing with large QAP instances. In addition, the research can be done on using Physarum-based Metaheuristic with the other optimisation problems. Some attempts to solve TSP are presented in Appendix \ref{chapter:tsp}, however the results are preliminary and should be extended with a further research.
