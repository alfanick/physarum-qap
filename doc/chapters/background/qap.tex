\section{Quadratic Assignment Problem}
\label{section:background_qap}

\subsection{History and importance with examples (2)}

The Quadratic Assignment Problem is a great challenge in combinatorial optimization.
It history starts with Tjalling Koopmans and Martin Beckmann, who presented a book named \textit{Assignment Problems and the Location of Economic Activities} in 1957 \cite{koopmans-beckmann1957}.
They have focused on allocation of indivisible resources using assignment of plants to location as an example.
Two problems were considered, first in which the transportation costs between plants could be ignored and second where the costs are included.

Ignoring the costs presents a relatively simple problem.
The task is to assign into pairs two sets of an eual number $n$ of similar elements.
Each assignment has a different score and after choosing all elements the sum of scores is calculated.
The objective is to achieve the highest result and such dilemma is named linear assignment problem.

Companies often have different tasks with different difficulties for their workers.
Each employee could do the job more or less the with the same result, which will be the score of the pairing.
Picking the best person for the job could be a good example of this simple problem.
However, this expect that each person is suitable to do the job alone.
Adding that tasks depend on each other makes selection less trivial.

Introducing an assumption that some elements from one set are dependent on another elements from the same set complicates greatly linear assignment problem and presents quadratic assignment problem.
In this situation we have two sets as before and a matrix with weights, where each matrix's element respresents one combination of assignment.

\subsection{Math definition (1)}


\subsection{Known algorithms (11-16)}

\subsubsection{Exact solutions (2)}
\subsubsection{Linearizations (3-5)}

\subsubsection{Metaheuristics (6-8)}

\paragraph{Greedy (1)}
\paragraph{Tabu search (2)}
\paragraph{Simulated annealing (2)}
\paragraph{Ant colony (2-3)}

