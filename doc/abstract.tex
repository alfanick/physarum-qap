\chapter*{Abstract}

The Quadratic Assignment Problem is an optimisation problem with many practical use cases, however, its solutions are usually approximated. It is a representant of the NP-hard computation class, therefore there is no algorithm known for solving this problem in a polynomial time. Even for small instances, the computation of the exact assignment can take too much time, which makes it unpractical. This creates a need for various heuristical methods for approximating the solution, which can be used in fields of logistics, computer aided design or even molecular chemistry.

Recently, an emerging computational behaviour of \textit{Physarum polycephalum} is a subject for thorough research. Many works have described this slime mould as useful in various branches of the computing science. The plasmodial form of \textit{Physarum polycephalum} is used to solve a maze problem, provide a design for a robust network or approximate the Travelling Salesman Problem. Moreover, the slime mould is even mathematically modelled and simulated on various levels.

In context of these observations, we fuse the knowledge of both fields: QAP and \textit{Physarum polycephalum} into an algorithm that leads to an approximation of Quadratic Assignment Problem. Observations of a living plasmodium allowed us to use some of its behaviour in a design of the Physarum-based Metaheuristic. This novel method of looking through the search space is used to solve the QAP.

As a part of this thesis an implementation of the proposed algorithm is given. It is thoroughfully tested and compared to the already existing methods. Our method provides useful solutions for many practical use cases, thus proving the suitability for such sophisticated problems. Furthermore, it can be tuned to solve complex instances of QAP or even used for approximating other optimisation problems.
