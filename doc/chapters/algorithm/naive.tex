\section{Naive Space Search using Physarum Polycephalum}
\label{section:algorithm_naive}

Using classic definition of Quadratic Assignment Problem, we can assign a cost $c : f \rightarrow \mathbb{R}$, $c(f) = \sum_{a,b\in P}w(a,b)\cdot d(f(a), f(b))$ for each assignment $f$. The goal is to find assignment minizing the cost. To approximate the solution, we propose variation of brute force algorithm implemented as a Physarum Machine.

% TODO move equation to new line
The input transformation is rather simple: for each possible $n!$ assignments we compute its cost and transform it to size of food source using examplar function $g(c(f)) = \frac{a}{q^{c(f)+k}}+b$, where $a$ and $k$ are scaling factors and $b$ is bias, the exponential function of base $q$ is used to amplify small costs. The parameters of function $g$ can be selected in such manner that small difference in cost is represented by not-so-small difference in food source mass. Food sources of weight proportional to $g(c(f))$ made of porridge (sterile oatmeal paste) are placed uniformly on the substrate --- for each assignment there is a respective food source with its size exponentially inversely proportional to the cost. Number of the slime mould colonies in plasmodial stage are placed on this prepared environment. Now observations could be made --- the experiment proceeds as long as plasmodium actively moves within given observation timeframe. 

Exploiting the fact that \textit{Physarum Polycephalum} prefers to consume the biggest food source, an output transformation simply takes position of the plasmodium and returns an assignment linked with physical food source where the plasmodium resides. It should be remembered that the slime mould is a living creature and results obtained using this algorithm are just an approximation as plasmodium behaves nondeterministically when foraging.

Presented algorithm is just a concept and has not been tested in a wet lab as it is highly unpractical. The input transformation requires computing food size for every possible $n!$ assignment, then a human or CNC must place this food on a substrate (which itself must be big enough to contain everything), then many days or even weeks of observations must be done until the plasmodium stops crawling --- just for a single problem definition. With that much work, we would get only an approximation --- the approximation of $argmin$ function which could have been worked out using a computer giving accurate results, probably in time shorter than time needed for computing the input transformation. Usage of such physarum machine gives no additional quality --- obtained result is just an approximation calculated in a longer time than basic exhausitive brute force which gives an optimal assignment. While it was an educational experience, we conclude that we are not able to create a physarum machine for solving QAP working in reasonable time with reasonable results.
