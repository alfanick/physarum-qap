\section{Quadratic Assignment Problem}
\label{section:background_qap}

\subsection{History and importance with examples (2)}
The Quadratic Assignment Problem is an enormous challenge in combinatorial optimization.
Its history starts with Tjalling Koopmans and Martin Beckmann, who presented a book named \textit{Assignment Problems and the Location of Economic Activities} in 1957 \cite{koopmans-beckmann1957}.
They have focused on the allocation of indivisible resources using the assignment of plants to location as an example.
Two problems were considered, first in which the transportation costs between plants could be ignored and second where the costs are included.

Since the presentation of the first formulation, there were no dramatic advances in improvement of solving methods.
Ignoring the prices shows a relatively simple problem.
The task is to assign into pairs two sets of an equal number $n$ of similar elements.
Each assignment has a different score and after choosing all elements, the sum of scores is calculated.
The objective is to achieve the highest result.
Such dilemma is named linear assignment problem.

Companies often have many tasks with different difficulties for their workers.
Each employee could do the task more or less with the likewise result, which will be the rate of the pairing.
Picking the most fitting person for the specific duty could be a valid example of this simplistic problem.
However, this expects that each person is suitable to do the activity alone.
Adding that tasks hinge on each other makes selection less trivial.

Introducing an assumption that some elements from one set are dependent on other elements from the same set complicates greatly linear assignment problem and presents quadratic assignment problem.
In this situation, we have two sets as before and a matrix with weights, where each matrix's element represents one combination of assignment.
Here, choosing the solution for even a small number of elements could be demanding.

Nonetheless, the problem's importance could be shown by real life patterns existing in many fields of studies.

In production, assigning different factories to locations is one of them.
Plenty of companies need to divide responsibilities between each facility, so they could create various parts of a product, which would be later assembled together at the main branch.
The factories need to communicate constantly and allow movement of items between them.
Here, the essential factor is the distance.
The most dependent branches should be closer to each other in order to fasten production.
It fits perfectly the presented dilemma because minimizing the total cost of transportation could be achieved only by allocating items smartly.
It also matches the model - there are two sets, locations, and factories, with an equal number of elements and facilities dependent on others.

In electronics, the backboard wiring problem could be an example.
Placement of distinctive elements on the backboard is not a trivial task.
Items contingent on others creating a net connected by wire.
The longer the coil, the resistance is bigger.
This affects the performance of the whole electrical network and minimization is indicated.

In the service industry, an example could be locating distinct departments.
Similar to factories and location, here, offices and floors or rooms are allocated.

\subsection{Math definition (1)}
Let m be positive integer and $M = { 1, 2, ..., m }$. The Quadratic Assignment Problem (QAP) can be formulated as
\begin{align}
  \text{maximize} &\null \sum_{i \in M} \sum_{j \in M} \sum_{p \in M} \sum_{q \in M} a_{ijpq}x_{ij}x_{pq} + \sum_{i \in M} \sum_{j \in M} b_{ij} x_{ij} \\
  \text{subject to} &\null \sum_{i \in M} x_{ij} = 1, \; j \in M \\
  &\null \sum_{j \in M} x_{ij} = 1, \; i \in M \\
  &\null x_{ij} = 0 \; or \; 1, \; i, j \in M
\end{align}

This formulation is the most generalized model of quadratic assignment problem.
It could have been defined differently, however, this shows the objective function as a quadratic one, which clearly indicated the name.
The discrete nature and the complexity of quadratic objective function cause difficulties in finding an optimal solution thus making the problem NP-hard.
Horowitz and Sahni have proved that in their book \cite{horowitz1978fundamentals} by showing that QAP needs an exponential time algorithm to solve optimally.


\subsection{Known algorithms (11-16)}
\subsubsection{Exact solutions (2)}

The most simple method of resolving QAP is by using an enumeration or populary named "branch and bound", which is to evaluate all of the $m!$ permutations (assignments) and recording the best solution.
Every calculation must be done from the start.
This requires $m^2$ steps and must be computed $m!$ times, which makes it useless for bigger instances.
However, when the problem is small, it has an advantage due to simplicity - the implementation is simple and requires little memory.

Surprisingly, also the time could be better. Franzer in his paper \cite{frazer1997602} presented that for test problem with $m=10$ the enumeration could be faster than linear programming. Comparing memory usage also gave amazing results thus proving that linear or 0-1 programming are not appropriate solution methods.

\subsubsection{Linearizations (3-5)}
A great approach to resolving such complex problem is to reduce it to a linear programming problem.
Unfortunately, most of the time this could be impractical even for small m.
The first to transform the problem was Lawler, who represented it by an equivalent 0-1 linear programming problem and facilities the computations \cite{lawler1963}. He based his thinking on a traveling salesman problem looking for similarities between these two \cite{charnsethikul1988exact}.

Defining $m^4$ variables $y_{ijpq}$ as $m^2$ variables $x_{ij}$ where
\begin{equation}
y_{ijpq} = x_{ij}x_{pq}
\end{equation}
the problem can be linearized and stated as
\begin{align}
  \text{minimize} &\null \sum_{i \in M} \sum_{j \in M} \sum_{p \in M} \sum_{q \in M} a_{ijpq}y_{ijpq} + \sum_{i \in M} \sum_{j \in M} b_{ij} x_{ij} \\
  \text{subject to} &\null \sum_{i \in M} x_{ij} = 1, \; j \in M \\
  &\null \sum_{j \in M} x_{ij} = 1, \; i \in M \\
  &\null x_{ij} = 0 \; or \; 1, \; i, j \in M \\
  &\null \sum_{i \in M} \sum_{j \in M} \sum_{p \in M} \sum_{q \in M} y_{ijpq} = m^2 \; \\
  &\null x_{ij} + x_{pq} - 2y_{ijpq} \leq 0, \; i, j, p, q \in M \\
  &\null y_{ijij} = 0 \; or \; 1, \; i, j, p, q \in M \\
\end{align}
In result there are $m^4 + m^2$ binary variables and $m^4 + 2 m^2 + 1$ constraints.
His definition has still been impractical for any bigger m.

Kaufmann and Broeckx created another linearization, which has the smallest number of variables and constraints \cite{kaufman1978algorithm} and is based on the method suggested by Glover.
Such manipulation reduces the complexity of the problem thus making calculations faster.
The authors rearrange the objective function into
\begin{equation}
  \sum_{i \in M} \sum_{j \in M} x_{ij} \sum_{p \in M} \sum_{q \in M} a_{ijpq} x_{pq}
\end{equation}

Afterwards they define $m^2$ new real variables
\begin{equation}
  w_{ij} := x_{ij}\sum_{p \in M} \sum_{q \in M} a_{ijpq} x_{pq}
\end{equation}

Additionaly introducing $m^2$ constraints $ c_{ij} := \sum_{p \in M} \sum_{q \in M} a_{ijpq} $ the problem could be formulated as
\begin{align}
  \text{minimize} &\null \sum_{i \in M} \sum_{j \in M} w_{ij}\\
  \text{subject to} &\null \sum_{i \in M} x_{ij} = 1, \; j \in M \\
  &\null \sum_{j \in M} x_{ij} = 1, \; i \in M \\
  &\null x_{ij} = 0 \; or \; 1, \; i, j \in M \\
  &\null c_{ij}x_{ij} + \sum_{p \in M} \sum_{q \in M} a_{ijpq}x_{pq} - w{ij} \geq c_{ij} = 0 \; i, j \in M \\
  &\null w_{ij} \geq 0, \; i, j \in M \\
\end{align}

In result there are $m^2$ real variables, $m^2$ binary variables and $m^2 + 2m$ constraints.
This improves slightly previous methods.
However, Kaku and Thompson reported \cite{kaku1986exact} that for small sizes, such as m = 8, the requirements become too large and optimality could not be proved.

Another linearization was presented by Frieze and Yadegar \cite{frieze1983quadratic}.
Obtaining their mixed integer linear programming formulation could be achieved by replacing the product of binary variables by continuous variables as
\begin{equation}
  y_{ijpq} := x_{ij} x_{pq}
\end{equation}

The outcome definition being
\begin{align}
  \text{minimize} &\null \sum_{i \in M} \sum_{j \in M} \sum_{p \in M} \sum_{q \in M} a_{ijpq}y_{ijpq} + \sum_{i \in M} \sum_{j \in M} b_{ij} x_{ij} \\
  \text{subject to} &\null \sum_{i \in M} x_{ij} = 1, \; j \in M \\
  &\null \sum_{j \in M} x_{ij} = 1, \; i \in M \\
  &\null x_{ij} = 0 \; or \; 1, \; i, j \in M \\
  &\null \sum_{i \in M} y_{ijpq} = x_{pq}, \; j, p, q \in M \\
  &\null \sum_{j \in M} y_{ijpq} = x_{pq}, \; i, p, q \in M \\
  &\null \sum_{p \in M} y_{ijpq} = x_{ij}, \; i, j, q \in M \\
  &\null \sum_{q \in M} y_{ijpq}= x_{ij}, \; i, j, p \in M \\
  &\null y_{ijij} = x_{ij}, \; i, j \in M \\
  &\null 1 \geq y_{ijpq} \geq 0, \; i, j, p, q \in M
\end{align}

In result there are $m^4$ real variables, $m^2$ binary variables and $m^4 + 4m^3+m^2+2m$ constraints.

Adams and Johnson presented also a new 0-1 linear integer programming formulation \cite{adams1994improved}.
It has similarities to the previous definition.
\begin{align}
  \text{minimize} &\null \sum_{i \in M} \sum_{j \in M} \sum_{p \in M} \sum_{q \in M} a_{ijpq}y_{ijpq} \\
  \text{subject to} &\null \sum_{i \in M} x_{ij} = 1, \; j \in M \\
  &\null \sum_{j \in M} x_{ij} = 1, \; i \in M \\
  &\null x_{ij} = 0 \; or \; 1, \; i, j \in M \\
  &\null \sum_{i \in M} y_{ijpq} = x_{pq}, \; j, p, q \in M \\
  &\null \sum_{j \in M} y_{ijpq} = x_{pq}, \; i, p, q \in M \\
  &\null y_{ijpq} = x_{pqij}, \; i, j, p, q \in M \\
  &\null y_{ijpq} \geq 0, \; i, j, p, q \in M
\end{align}

In result there are $m^2$ binary variables, $m^4$ continuous variables and $m^4+2m^3+2m$ constraints.

\subsubsection{Metaheuristics (6-8)}
Methods presented previously are not efficient with bigger instances.
In order to enhance the results, heuristics were used.
These approximate solution techniques have been used to facility combinatorial problems.
After it became clear, that many of dilemmas are NP-hard, the role of heuristics increased.
Even thought problems of dimensions $m > 20$ are still impractical to resolve due to high computational time, heuristics slightly improved a situation for medium size cases.


\paragraph{Greedy (1)}
\paragraph{Tabu search (2)}

In the beginning of heuristics, the most popular approach was Local Search.
Starting from one solution, the algorithm was improving it by applying slight modifications.

Tabu search is a metaheuristic local search method created by Glover in 1986 \cite{glover1986future}, which improves the problems encountered in Local Search.
Its goal is to conquer local optimality using not only advancing the result but also worsening.
It is achieved by remembering already visited outcomes, which helps to see promising directions for further exploration.

The method has few main elements, which needs to be defined in order to understand the procedure:
\begin{itemize}
  \item A \textit{search space} is a space of all possible solutions that could be visited during the search. This includes all feasible answers of the problem.
  \item A \textit{neighborhood structure} is a subset of search space, which could be accomplished by applying a single \textit{transformation} to a current solution.
  \item A \textit{move} is an operation, which changes solution $x$ to a neighbor $x'$. In the case of QAP, the neighborhood could be created by exchanging one pair of the solution.
  \item A \textit{tabu list} is a list containing forbidden (tabu) moves and it is updated constantly with each iteration. It consists of the records of the recent history of transformations and prevents cycling back to already visited result.
  \item An \textit{aspiration criterion} is a condition associated with each tabu move, which permits canceling its tabu state. A most common example is allowing the tabu move if the value of the objective function is superior to current. This obviously means that the solution was not previously visited.
\end{itemize}

In the beginning, the feasible solution is chosen and the tabu list is updated.
There is also a list of neighbors available.
The algorithm chooses the best neighboring solution, which could be produced by a non-tabu move.
The neighbor does not have to raise the value of objective function.
Afterward, the neighbor is the current result and the algorithm goes from the start.
The calculations are ended after \textit{termination criteria}, which could be: fixed number of iterations, the number of iterations without an improvement in the value, after reaching the threshold value.

In order to prevent visiting the same solutions over and over, a tabu criterion is used.
It identifies moves, which could lead to cycles, and adds them to the tabu list.
However, such moves could be applied if the aspiration criterion shows that they could introduce potentially winning solutions.
The tabu list is an essential change comparing to local search.

Each tabu search algorithm could be implemented slightly differently, which impede measurement of performance.
Currently, three different tabu search methods were proposed.
First, a fixed tabu list proposed by Skorin-Kapov \cite{skorin1990tabu}.
The size of tabu list is a parameter, which needs to tunned before starting the calculation.
It can also be dependent on the specific instance, thus such method is not recommended.
Second, a robust tabu search proposed by Taillard \cite{taillard1991robust}.
Here, a maximum and a minimum are defined and the size of the tabu list is chosen among them.
Third, a reactive tabu search proposed by Battiti and Tecchiolli \cite{battiti1994reactive}.
The size of the tabu list is adapting to the current situation.
When the cycle is identified, the size is increased according to the length of the repetition.
This approach shows the best results for QAP.

\paragraph{Simulated annealing (2)}
\paragraph{Ant colony (2-3)}

