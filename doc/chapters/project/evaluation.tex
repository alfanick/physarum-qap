\section{Performance evaluation}
\label{section:project_evaluation}

The implementation of Physarum-based Metaheuristic algorithm is tested under variety of conditions. Some tests have been made to show general behaviour of the algorithm depending on input problem size, while other test influence of various parameters, helping out with their selection. In the end the algorithm is compared to other common algorithms.


\subsection{Dataset description}

The algorithm is tested on QAPLIB dataset library \cite{burkard1997qaplib}. This dataset is de~facto standard for testing various Quadratic Assignment Problem solvers. It includes dozens of problem definitions with optimal solutions (or approximations where no optimal solution has been found yet) in unified format. Some of the inputs are synthetic, generated using various algorithms (i.e. \texttt{lipa} or \texttt{nug}), while some problems are taken from the real world (i.e. \texttt{els} or \texttt{ste}).


\subsection{Testing methodology}

Same seed value has been used across the tests, unless mentioned otherwise --- this simple trick ensures the same initial position even if different configuration options are used. Test cases \texttt{lipa20a} and \texttt{lipa90a} are used for testing various parameters as they greatly differ in size ($n=20$ and $n=90$), so influences of the parameters can be observed. Values of observable parameters are measured every epoch (a discrete algorithm step), while execution time is bounded by 300~s limit, unless mentioned otherwise.

All tests have been done on the same test machine with given specification: processor Intel Xeon E3-1246 3.5GHz, DDR3-1600 32GB RAM, Linux kernel 3.19. Every plot used in this work is generated automatically from logs made by \texttt{physarum-debug} (the logs are included with the source). Some detailed numerical values are provided in tables in Appendix \ref{chapter:results}.


\subsection{Colony initialization phase}

The algorithm initialization phase begins with sampling of $k$ random solutions from whole $n!$ search space --- than on $l$ best solution instances of \texttt{Plasmodium} are put. Size of population directly affects total number of visited solutions (solutions that were used as food source when plasmodia crawled), larger populations visit more solutions (figure \ref{figure:am_visited_solutions}), because they start in multiple points of the space search. An obvious observation is that introducing more plasmodia results in more computation, as a result each discrete epoch takes longer time to compute --- these experiments were limited to 300~s execution time, larger colonies finished on earlier epochs.

Larger colonies are merging earlier than smaller colonies (figure \ref{figure:am_state_of_colony}), which is an anticipated behaviour --- the chance of encountering another plasmodium is greater as there are more plasmodia sharing the same space search. It can be observed that as a result of merging number of solutions stops increasing drastically as in early epochs, because there are less plasmodia crawling.

\begin{figure}
  \centering

  \begin{subfigure}{\textwidth}
    \includegraphics[width=1.1\textwidth,center]{algorithm/metaheuristic/charts/single/lipa20a/total_food_eaten_count - lipa20a $l=300$ $k=300$.\eop}
    \caption{instance \texttt{lipa20a} $n=20$}
  \end{subfigure}
  \par\bigskip
  \begin{subfigure}{\textwidth}
    \includegraphics[width=1.1\textwidth,center]{algorithm/metaheuristic/charts/single/lipa90a/total_food_eaten_count - lipa90a $l=300$ $k=300$.zoomed.\eop}
    \caption{instance \texttt{lipa90a} $n=90$}
  \end{subfigure}
  
  \caption{Number of visited solutions with different colony size $l$, samples $k$}
  \label{figure:am_visited_solutions}
\end{figure}

\begin{figure}
  \centering

  \begin{subfigure}{\textwidth}
    \includegraphics[width=1.1\textwidth,center]{algorithm/metaheuristic/charts/single/lipa90a/state of colony - lipa90a $l=10$ $k=30$.\eop}
    \caption{population $l=10$, samples $k=30$}
  \end{subfigure}
  \par\bigskip
  \begin{subfigure}{\textwidth}
    \includegraphics[width=1.1\textwidth,center]{algorithm/metaheuristic/charts/single/lipa90a/state of colony - lipa90a $l=100$ $k=300$.\eop}
    \caption{population $l=100$, samples $k=300$}
  \end{subfigure}
  \par\bigskip
  \begin{subfigure}{\textwidth}
    \includegraphics[width=1.1\textwidth,center]{algorithm/metaheuristic/charts/single/lipa90a/state of colony - lipa90a $l=300$ $k=300$.\eop}
    \caption{population $l=300$, samples $k=300$}
  \end{subfigure}

  \caption{State of colony with different colony size $l$, samples $k$ (instance \texttt{lipa90a} $n=90$)}
  \label{figure:am_state_of_colony}
\end{figure}


\begin{figure}
  \centering

  \begin{subfigure}{\textwidth}
    \includegraphics[width=1.1\textwidth,center]{algorithm/metaheuristic/charts/single/lipa20a/frontier_best_cost aggregated min - lipa20a $l=300$ $k=300$.zoomed.\eop}
    \caption{instance \texttt{lipa20a} $n=20$}
  \end{subfigure}
  \par\bigskip
  \begin{subfigure}{\textwidth}
    \includegraphics[width=1.1\textwidth,center]{algorithm/metaheuristic/charts/single/nug20/frontier_best_cost aggregated min - nug20 $l=300$ $k=300$.zoomed.\eop}
    \caption{instance \texttt{nug20} $n=20$}
  \end{subfigure}
  
  \caption{Cost of the best detected solution by any plasmodium with different colony size $l$, samples $k$}
  \label{figure:am_best_cost}
\end{figure}

For small testcases a regularity has been observed --- larger populations tend to find better results in earlier epochs (figure \ref{figure:am_best_cost}). Furthermore, larger $\frac{l}{k}$ proportions (number of plasmodia vs number of samples) are preferred as they exhibit the same characteristic --- when the same number of samples is used, more plasmodia started on these samples yield better results earlier. Plasmodium initialized on a bad solution can crawl towards better one, where a plasmodium initialized on a relatively good solution may not find any better in its local neighbourhood. Thus one should use $l$ close to $k$ to fully use the algorithm potential. 

% TODO different energies

% TODO scale/base
% TODO initial
% TODO special cases - random walk and dead plasmodium

% TODO store global minimum

% TODO time/memory
% TODO complete run for qaplib
% TODO comparison with other
