\section{Goal}
\label{section:introduction_goal}

This thesis presents the road to solving Quadratic Assignment Problem (QAP) based on behaviour of Physarum machines.
Deep analysis of each part of the main dilemma is essential to proceed with future works.

The first task is to carry out the detailed investigation of the behaviour and capabilities of the slime mould. Without the understanding of organisms, it is not possible to replicate its operations. For this purpose, the living plasmodia will be observed, giving us details of the schemes of the movement when looking for food. This will be studied in order to extract similar patterns and facilitate the creation of
calculation method based on them, which could be transported into the computer environment. We will implement such plasmodial behaviour for solving QAP.

Furthermore, not only the direct observation of \textit{Physarum polycephalum} behavior is needed here, but also a careful examination of previous studies. Such research will also give as a thorough look into behaviour and abilities of the slime mould, especially taking into account the possibility of using our model algorithm as an inspiration for solving computational problems.

Next, the analysis of the research related to the QAP in general will be conducted, leading to better understanding of the problem and showing the current practices for solving it. Recognising the dilemma will make it easier to fit the algorithm based on slime moulds to the QAP.

The key element of this thesis is to apply methods used by \textit{Physarum polycephalum} for solving QAP. This step will consist of adaptation of the biological mechanisms, implementation of simulation and analysis of the results. In this way, we connect the previously acquired theoretical knowledge and a practical task. And last, but not least, our aim will be to create the innovative method for solving QAP.
