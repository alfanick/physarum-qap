\section{Motivation}
\label{section:introduction_motivation}

The motivation for this thesis was our interest in topics related to computing science, but also to the world around us. The behavior of \textit{Physarum}, which is often compared to a simple machine, creates many opportunities to unveil a biological side of computeting science, thus making the topic fascinating.

Nowadays, scientists put great emphasis on discovering and analysing the nature, often improving the world surrounding us. Generally, two ways of development of this field of study could be distinguished. The first one focuses on improving the biological flaws of humans and animals. A good case is a field of studies related to the creation of natural prosthetics, making life easier for the people without limbs. The second one is a transmission of the known naturally patterns to the computation environment. Observing the nature leads to organic algorithms, usage of which can solve issues seemingly unrelated to initial problem and often gives much better results than traditional computer science approach. In example, thanks to such research, the ant algorithm was implemented, which uses the behaviour of an ant colony searching for the best path between their home and food. These unconventional methods of computation provides excellent results for hitherto very complicated mathematical problems.

The slime moulds have great potential for algorithmic design in many fields. Until now, several studies linked to these organisms were conducted, though, details remain a mystery even to experts. One can give an example of experiment, carried on slime moulds, which was solved the maze problem. The organism found the shortest path between two food soruces in the environment with walls, thus finding the solution for the maze. More detailed description and more usecases are presented in the following chapter of this thesis. Nonetheless, these interesting achievements were the reason behind the choice of the subject.

On the other hand, the thesis focuses on the Quadratic Assignment Problem, which is alone a challenging topic. It can be thought as reflecting the real difficulties faced by the managers of logistics companies or electronics designers. They prefer optimal results, however, the complexity of this dilemma makes it almost impossible to find such in a reasonable time. This demand urges scientists to explore this issue further and try to look for a new ways of solving the problem. 

The QAP could be a great challenge for inventing a new unconventional algorithm based on the behavior of physarum. 
